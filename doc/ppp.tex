\documentclass{article}

\usepackage{amsmath}
\usepackage{amssymb}
\usepackage{listings}
\usepackage{mathabx}
\usepackage{textcomp}

\usepackage[footskip=0.5in, top=0.75in, bottom=1in, left=1.5in, right=1.5in]{geometry}

\newenvironment{Table}
  {\begin{center}\begin{tabular}{l @{\;\;} l @{\;\;} l @{\quad\quad} l}}
  {\end{tabular}\end{center}}
\newenvironment{BigTable}
  {\begin{center}\begin{tabular}{
    l @{\;\;} l @{\;\;} l
    @{\quad\quad\quad\quad}
    l @{\;\;} l @{\;\;} l
  }}
  {\end{tabular}\end{center}}

\newcommand{\layout}[2]{\textit{layout}(#1, #2)}
\newcommand{\resolve}[2]{\textit{resolve}(#1, #2)}

\newcommand{\ind}[2]{\texttt{#1 :>> #2}}
\newcommand{\cat}[2]{\texttt{#1 :+ #2}}
\newcommand{\choice}[2]{\texttt{#1 :| #2}}

\newcommand{\txt}[1]{\texttt{Text #1}}
\newcommand{\nil}{\texttt{Nil}}
\newcommand{\err}{\texttt{Err}}
\newcommand{\nl}{\texttt{Newline}}
% \newcommand{\ind}[2]{\texttt{Indent #1 #2}}
\renewcommand{\flat}[1]{\texttt{Flat #1}}
% \newcommand{\cat}[2]{\texttt{Concat #1 #2}}
% \newcommand{\choice}[2]{\texttt{Choice #1 #2}}

% \newcommand{\txt}{T}
% \newcommand{\nil}{{\epsilon}}
% \newcommand{\err}{{!}}
% \newcommand{\nl}{{\dlsh}}
% \newcommand{\ind}[2]{#1 \Rightarrow #2}
% \renewcommand{\flat}[1]{\lfloor #1 \rfloor}
% \newcommand{\cat}[2]{#1 + #2}
% \newcommand{\choice}[2]{#1 \,|\, #2}
\newcommand{\spaces}[1]{\texttt{replicate}\;#1\;\texttt{\textquotesingle \textquotesingle}}

%\newcommand{\doubleplus}{\mathbin{+\mkern-8mu+}}

\newcommand{\code}[1]{\texttt{#1}}

\lstset{
  language=Haskell,
  basicstyle=\ttfamily,
  mathescape
}

\begin{document}

\author{Justin Pombrio}
\title{Partial Pretty Printing}
\maketitle

[FILL] introduction; relationship to Wadler's Prettier Printer and others

[FILL] only consider length of first line, trading off expressiveness for efficiency. Linear time.

[FILL] peephole efficiency

\section{Reference Implementation}

[FILL]

\subsection{Documents}

Documents will be made by eight constructors:

\begin{lstlisting}
data Doc = Nil         -- empty document
         | Text String -- text (without newlines)
         | Newline     -- newline
         | Flat x      -- disallow newlines in x
         | i :>> x     -- indent x by i spaces
         | x :+ y      -- concatenation of x and y
         | x :| y      -- choice between x and y
         | Err         -- error
\end{lstlisting}

$\nil$ prints nothing at all, $\txt{t}$ prints a string verbatim, and $\nl$ prints a newline.
$\cat{x}{y}$ is the \emph{concatenation} of $x$ and $y$: it first prints $x$, then $y$, with no
separation.

Parts of a document can be indented. $\ind{i}{x}$ indents $x$ by $i$ spaces (specifically, it
inserts $i$ spaces after every newline in $x$.) $\flat{x}$, on the other hand, \emph{disallows}
newlines: it produces an error $\err$ if it encounters a newline in $x$. This is only useful in
conjunction with choices, described next.

A choice $\choice{x}{y}$ will print either $x$ or $y$. Choices are resolved \emph{greedily}, and
only based on the current line. Specifically, $x$ is chosen iff it does not cause the current line
to exceed the pretty printing width or choosing $y$ would result in an error.

[TODO: move para?] This rule for resolving choices does limit how pretty documents can be. Certain
ways of formatting can be either great or terrible, and you cannot tell which just by looking at
their first line.  These must be avoided, lest someone runs into the terrible case. [FILL: examples,
e.g. alignment].  However, this way of resolving choices also has a very strong advantage: it allows
for partial pretty printing. [REF: more details].

For example, this document:
\begin{lstlisting}
2 :>> (Text "Hello" :+ (Text " " :| Newline) :+ "world")
\end{lstlisting}
will either print on one line if the printing width is at least 11:
\begin{lstlisting}
Hello world
\end{lstlisting}
on on two lines otherwise:
\begin{lstlisting}
Hello
  world
\end{lstlisting}

\subsection{Pretty Printing without Choices}

Let's turn this informal definition into code. For the moment, we'll keep the implementation naive
at the expense of efficiency (even to the extent of exponential inefficiency). We will later use it to
derive a more efficient implementation that provably behaves the same.

First we need a data structure for the output of pretty printing, and some basic operations on it.
The output---call it a \emph{layout}---is either a list of lines, or an error in case of
encountering $\err$:

\begin{lstlisting}
data Layout = LErr | Layout [String]
\end{lstlisting}

Layouts can be displayed, flattened, indented, and appended:

\begin{lstlisting}
display :: Layout -> String
display LErr = "Error!" -- convenient for testing
display (Layout lines) = intercalate "\n" lines

flatten :: Layout -> Layout
flatten LErr = LErr
flatten (Layout lines) =
  if length lines == 1
  then Layout lines
  else LErr

indent :: Int -> Layout -> Layout
indent _ LErr = LErr
indent i (Layout (line : lines)) =
  Layout (line : map addSpaces lines)
  where addSpaces line = replicate i ' ' ++ line

append :: Layout -> Layout -> Layout
append (Layout lines1) (Layout lines2) =
  Layout (init lines1 ++ [middleLine] ++ tail lines2)
  where middleLine = (last lines1 ++ head lines2)
append _ _ = LErr
\end{lstlisting}

Pretty printing a document that contains no choices is completely straightforward:

\begin{lstlisting}
-- Preliminary!
pp :: Doc -> Layout
pp Err = LErr
pp Empty = Layout [""]
pp (Text t) = Layout [t]
pp Newline = Layout ["", ""]
pp (Flat x) = flatten (pp x)
pp (i :>> x) = indent (pp x)
pp (x :+ y) = append (pp x) (pp y)
\end{lstlisting}

In fact, since we aren't making any choices, notice that we aren't even using the pretty printing
width yet.

\subsection{Pretty Printing with Choices}

Ideally, pretty printing would recursively walk the document and resolve any choices as it
encountered them. However, this is not easy because at that point we have not printed what comes
after the choice, which might effect how long the current line is. Instead, let's delay resolving
any choices until we have processed the whole document. Thus \code{pp} will produce a decision tree
of \code{Layout}s, with a branch for each \code{choice} and storing the line number at which the
choice is to be made:

\begin{lstlisting}
data Layouts = Branch Int Layouts Layouts
             | Leaf Layout
\end{lstlisting}

We need to extend our previous functions to handle \code{Layouts} as well:
\begin{lstlisting}
flatten' :: Layouts -> Layouts
flatten' (Branch n x y) = Branch n (flatten' x) (flatten' y)
flatten' (Leaf layout) = Leaf (flatten layout)

indent' :: Int -> Layouts -> Layouts
indent' i (Branch n x y) = Branch n (indent' i x) (indent' i y)
indent' i (Leaf layout) = Leaf (indent i layout)

append' :: Layouts -> Layouts -> Layouts
append' (Branch n x y) z = Branch n (append' x z) (append' y z)
append' (Leaf layout) (Branch n x y) =
  Branch (n + numNewlines layout)
         (append' (Leaf layout) x)
         (append' (Leaf layout) y)
append' (Leaf layout1) (Leaf layout2) = Leaf (append layout1 layout2)
\end{lstlisting}
This is mostly a pointwise extension (i.e., boilerplate), except that if \code{append'} is given two
branches, see that it nests the right one into the left one rather than vice-versa. This represents
a decision to resolve choices from left to right.

The \code{pp} function is now trivial: it just calls the appropriate function:
\begin{lstlisting}
pp :: Doc -> Layouts
pp Err = Leaf LErr
pp Empty = Leaf (Layout [""])
pp (Text t) = Leaf (Layout [t])
pp Newline = Leaf (Layout ["", ""])
pp (Flat x) = flatten' (pp x)
pp (i :>> x) = indent' i (pp x)
pp (x :+ y) = append' (pp x) (pp y)
pp (x :| y) = Branch 0 (pp x) (pp y)
\end{lstlisting}

This produces a decision tree. We can resolve the decisions in the tree to get a single
\code{Layout} by simple recursion:
\begin{lstlisting}
resolve :: Int -> Layouts -> Layout
resolve _ (Leaf layout) = layout
resolve w (Branch n x y) =
  pick w n (resolve w x) (resolve w y)
\end{lstlisting}
where \code{pick} picks the "best" of two layouts. It avoids errors if possible, and otherwise picks
whichever layout does not exceed the maximum width \code{w} on the current line \code{n}:
\begin{lstlisting}
pick :: Int -> Int -> Layout -> Layout -> Layout
pick w _ LErr LErr = LErr
pick w _ (Layout lines) LErr = Layout lines
pick w _ LErr (Layout lines) = Layout lines
pick w n (Layout lines1) (Layout lines2) =
  if length (lines1 !! n) <= w
  then Layout lines1
  else Layout lines2
\end{lstlisting}

A user-facing interface should obtain a decision tree, resolve the decisions, and display the
resulting layout:
\begin{lstlisting}
pretty :: Int -> Doc -> String
pretty w x = display (resolve w (pp x))
\end{lstlisting}

This is a complete and correct implementation. However, it is exponential time, as it tries
resolving each choice both ways.

This makes it sound useless, but it is not! We can use it to prove algebraic laws about equivalence
between \code{Doc}s, and then use these laws to derive a more efficient implementation that (provably)
behaves the same.

\section{Laws of Pretty Printing}

This section lists equivalences between \code{Doc}s. An equivalence \code{x = y} between docs
typically means that \code{pp x = pp y}. However, for a couple of the laws [FILL] it instead means
that \code{x} and \code{y} are contextually equivalent in the expression \code{resolve w (pp C[x])},
for all widths \code{w} and contexts \code{C}.

All of the equivalences in this section can be proven from the implementation of \code{naivePP} using
basic equational reasoning. The proofs are given in [TODO: the appendix, but for now proofs\_tree.md].

\paragraph{Concatenation.}
The concatenation of two documents just prints them one after another (with no space or newline in
between). Thus concatenation with an empty document has no effect, and concatenation is associative.

\begin{align*}
  &\cat{\nil}{x} = \cat{x}{\nil} = x \tag{concat-unit} \\
  &\cat{(\cat{x}{y})}{z} = \cat{x}{(\cat{y}{z})} \tag{concat-assoc}
\end{align*}

\paragraph{Text.}
$\txt{t}$ is rendered exactly as is. Thus the empty string is the same as the empty document, and
the concatenation of two texts just concatenates their strings.

\begin{align*}
  \code{""} &= \nil \tag{text-empty} \\
  \cat{\txt{t1}}{\txt{t2}} &= \txt{(t1 ++ t2)} \tag{text-concat}
\end{align*}

\paragraph{Indentation and Flattening.}
Indentation and flattening can be lowered to the leaves of the document.  They both leave text (and
thus also empty documents) unchanged, but behave differently on newlines:
\begin{itemize}
\item A newline indented by $i$ is a newline followed by $i$ spaces.
\item The flattening of a newline is an error, since it's impossible to fit a newline on one line.
\end{itemize}

\begin{align*}
  \ind{i}{\nil} &= \nil
    \tag{indent-absorb-empty} \\
  \ind{i}{\txt{t}} &= \txt{t}
    \tag{indent-absorb-text} \\
  \ind{i}{\nl}  &= \cat{\nl}{(\spaces{i})}
    \tag{indent-newline} \\
  \ind{i}{(\cat{x}{y})} &= \cat{(\ind{i}{x})}{(\ind{i}{y})}
    \tag{indent-distr-concat} \\
  \ind{i}{(\choice{x}{y})} &= \choice{(\ind{i}{x})}{(\ind{i}{y})}
    \tag{indent-distr-choice} \\
  \\
  \flat{\nil} &= \nil
    \tag{flat-absorb-empty} \\
  \flat{(\txt{t})} &= \txt{t}
    \tag{flat-absorb-text} \\
  \flat{\nl} &= \err
    \tag{flat-newline} \\
  \flat{(\cat{x}{y})} &= \cat{\flat{x}}{\flat{y}}
    \tag{flat-distr-concat} \\
  \flat{\choice{x}{y}} &= \choice{\flat{x}}{\flat{y}}
    \tag{flat-distr-choice}
\end{align*}

Also, indentation respects addition: indenting by zero spaces is the same as not indenting at all,
and indenting by $i$ spaces and then $j$ spaces is the same as indenting by $i+j$ spaces. Along the
same lines, flattening twice has no effect.

\begin{align*}
  \ind{0}{x} &= x
    \tag{indent-identity} \\
  \ind{j}{(\ind{i}{x})} &= \ind{(i+j)}{x}
    \tag{indent-compose} \\
  \flat{(\flat{x})} &= \flat{x}
    \tag{flat-compose}
\end{align*}

\paragraph{Errors.}
There is only one source of error: flattening a newline (shown above). Once created, errors are
contagious and propagate up to the root of the document, except when inside a choice in which cae
they eliminate that option of the choice.

\begin{align*}
  \cat{\err}{x} = \cat{x}{\err} &= \err
    \tag{error-concat} \\
  \ind{i}{\err} &= \err
    \tag{error-indent} \\
  \flat{\err} &= \err
    \tag{error-flat} \\
  \choice{\err}{x} = \choice{x}{\err} &= x
    \tag{error-choice}
\end{align*}

\paragraph{Choice.}
Choice is associative, and if you concatenate a choice with some text, that's the same as
concatenating the text inside the choice. [FILL]

\begin{align*}
  \choice{x}{(\choice{y}{z})} &= \choice{(\choice{x}{y})}{z}
    \tag{choice-assoc} \\
  \cat{\txt{t}}{(\choice{x}{y})} &= \choice{(\cat{\txt{t}}{x})}{(\cat{\txt{t}}{y})}
    \tag{choice-distr-text-left} \\
  \cat{(\choice{x}{y})}{z} &= \choice{(\cat{x}{z})}{(\cat{y}{z})}
    \tag{choice-distr-right}
\end{align*}

\paragraph{A law that doesn't hold.}
Surprisingly, concatenation does \emph{not} in always distribute over choice. That is,
$\cat{x}{(\choice{y}{z})}$ is not always equal to $\choice{(\cat{x}{y})}{(\cat{x}{z})}$.  This is
because $x$ could contain a newline. In that case, we would have:

\[ \cat{\nl}{(\choice{y}{z})} =^? \choice{(\cat{\nl}{y})}{(\cat{\nl}{z})} \]

The document on the left makes a legitimate choice between $y$ and $z$, comparing their first lines
to see which one fits. However, the document on the right makes a degenerate choice: it compares the
first lines of $\cat{\nl}{y}$ and $\cat{\nl}{z}$, which are both empty! Thus it always picks
$\cat{\nl}{y}$, regardless of $y$ and $z$.  As a result, the concatenation of a newline and a choice
cannot be simplified.

\section{Efficient Implementation}

We can use these laws to derive an efficient implementation. The first step is to see that they are
sufficient to convert any document into a normal form, as follows:

\begin{itemize}
\item Eliminate \code{Nil} by converting it into text, using the ``text-empty" law.
\item Lower each \code{Flat} and \code{(:>>)} (indent) to the leaves of the document. If it hits a
newline, \code{Flat} will produce an \code{Err}, and \code{(:>>)} will insert spaces after the
newline. This uses all of the ``indent" and ``flat" laws except for indent-identity.
\item Raise all \code{Err}s up towards the root of the document. They will either eliminate an
option from a choice via the ``error-choice" law, or cause the entire document to become an
\code{Err}. This uses the four ``error" laws.
\end{itemize}

[FILL]

\end{document}
