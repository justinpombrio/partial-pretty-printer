\documentclass{article}

\usepackage{amsmath}
\usepackage{amssymb}
\usepackage{listings}
\usepackage{mathabx}
\usepackage{textcomp}

\usepackage[footskip=0.5in, top=0.5in, bottom=1in, left=1.5in, right=1.5in]{geometry}

\newenvironment{Table}
  {\begin{center}\begin{tabular}{l @{\;\;} l @{\;\;} l @{\quad\quad} l}}
  {\end{tabular}\end{center}}
\newenvironment{BigTable}
  {\begin{center}\begin{tabular}{
    l @{\;\;} l @{\;\;} l
    @{\quad\quad\quad\quad}
    l @{\;\;} l @{\;\;} l
  }}
  {\end{tabular}\end{center}}

\newcommand{\pp}[2]{\textit{pretty}(#1, #2)}
\newcommand{\layout}[2]{\textit{layout}(#1, #2)}
\newcommand{\resolve}[2]{\textit{resolve}(#1, #2)}

\newcommand{\txt}{T}
\newcommand{\nil}{{\epsilon}}
\newcommand{\err}{{!}}
\newcommand{\nl}{{\dlsh}}
\newcommand{\ind}[2]{#1 \Rightarrow #2}
\renewcommand{\flat}[1]{\lfloor #1 \rfloor}
\newcommand{\cat}[2]{#1 + #2}
\newcommand{\choice}[2]{#1 \,|\, #2}
\newcommand{\spaces}[1]{\texttt{replicate}\;#1\;\texttt{\textquotesingle \textquotesingle}}

\newcommand{\doubleplus}{\mathbin{+\mkern-8mu+}}

\begin{document}

\author{Justin Pombrio}
\title{Partial Pretty Printing}
\maketitle

\section{Introduction}

[FILL] introduction; relationship to Wadler's Prettier Printer and others

[FILL] only consider length of first line, trading off expressiveness for efficiency. Linear time.

[FILL] peephole efficiency

\subsection{Definition of Documents}

We'll have 

A \emph{document} is a set of instructions for rendering

My documents are similar to Wadler's, but more general in that choices are 

[FILL] description of documents

\begin{Table}
  $\textit{expr } x, y, z$
  &$::=$& $\nil$ & empty document \\
    &$|$& $\txt$ & text (without newlines) \\
    &$|$& $\nl$ & newline \\
    &$|$& $\ind{i}{x}$ & indent $x$ by $i$ spaces \\
    &$|$& $\flat{x}$ & flattening of $x$ \\
    &$|$& $\cat{x}{y}$ & concatenation of $x$ and $y$ \\
    &$|$& $\choice{x}{y}$ & choice between $x$ and $y$ \\
    &$|$& $\err$ & error
\end{Table}

\subsection{Laws for Documents}

\paragraph{Concatenation}
The concatenation of two documents just prints them one after another (with no space or newline in
between). Thus concatenation with an empty document has no effect, and concatenation is associative.

\begin{gather*}
  \cat{\nil}{x} = \cat{x}{\nil} = x \tag{concat-unit} \\
  \cat{(\cat{x}{y})}{z} = \cat{x}{(\cat{y}{z})} \tag{concat-assoc}
\end{gather*}

\paragraph{Text}
Text $\txt$ is rendered exactly as is. Thus the empty string is the same as the empty document, and
the concatenation of two texts just concatenates their strings.

\begin{align*}
  \texttt{""} &= \nil \tag{text-empty} \\
  \cat{\txt_1}{\txt_2} &= (\txt_1 \doubleplus \txt_2) \tag{text-concat}
\end{align*}

\paragraph{Indentation and Flattening}
Indentation and flattening can be lowered to the leaves of the document.  They both leave text (and
thus also empty documents) unchanged, but behave differently on newlines:
\begin{itemize}
\item A newline indented by $i$ is a newline followed by $i$ spaces.
\item The flattening of a newline is an error, since it's impossible to fit a newline on one line.
\end{itemize}

\begin{align*}
  \ind{i}{\nil} &= \nil
    \tag{indent-absorb-empty} \\
  \ind{i}{\txt} &= \txt
    \tag{indent-absorb-text} \\
  \ind{i}{\nl}  &= \cat{\nl}{(\spaces{i})}
    \tag{indent-newline} \\
  \ind{i}{(\cat{x}{y})} &= \cat{(\ind{i}{x})}{(\ind{i}{y})}
    \tag{indent-distr-concat} \\
  \ind{i}{(\choice{x}{y})} &= \choice{(\ind{i}{x})}{(\ind{i}{y})}
    \tag{indent-distr-choice} \\
  \\
  \flat{\nil} &= \nil
    \tag{flat-absort-empty} \\
  \flat{\txt} &= \txt
    \tag{flat-absort-text} \\
  \flat{\nl} &= \err
    \tag{flat-newline} \\
  \flat{\cat{x}{y}} &= \cat{\flat{x}}{\flat{y}}
    \tag{flat-distr-concat} \\
  \flat{\choice{x}{y}} &= \choice{\flat{x}}{\flat{y}}
    \tag{flat-distr-choice}
\end{align*}

Also, indentation respects addition: indenting by zero spaces is the same as not indenting at all,
and indenting by $i$ spaces and then $j$ spaces is the same as indenting by $i+j$ spaces.

\begin{align*}
  \ind{0}{x} &= x
    \tag{indent-identity} \\
  \ind{j}{(\ind{i}{x})} &= \ind{(i+j)}{x}
    \tag{indent-compose}
\end{align*}

\paragraph{Errors}
There is only one source of error: flattening a newline (shown above). Once created, errors are
contagious and propagate up to the root of the document, except when inside a choice in which cae
they eliminate that option of the choice.

\begin{align*}
  \cat{\err}{x} = \cat{x}{\err} &= \err
    \tag{error-concat} \\
  \ind{i}{\err} &= \err
    \tag{error-indent} \\
  \flat{\err} &= \err
    \tag{error-flat} \\
  \choice{\err}{x} = \choice{x}{\err} &= x
    \tag{error-choice}
\end{align*}

\paragraph{Choice}
Choice is associative, and if you concatenate a choice with some text, that's the same as
concatenating the text inside the choice. [FILL]

\begin{align*}
  \choice{x}{(\choice{y}{z})} &= \choice{(\choice{x}{y})}{z}
    \tag{choice-assoc} \\
  \cat{\txt}{(\choice{x}{y})} &= \choice{(\cat{\txt}{x})}{(\cat{\txt}{y})}
    \tag{choice-distr-text-left} \\
  \cat{(\choice{x}{y})}{z} &= \choice{(\cat{x}{z})}{(\cat{y}{z})}
    \tag{choice-distr-right}
\end{align*}

\subsection{Laws for Documents that Don't Hold}

Surprisingly, concatenation does \emph{not} in always distribute over choice. That is,
$\cat{x}{(\choice{y}{z})}$ is not always equal to $\choice{(\cat{x}{y})}{(\cat{x}{z})}$.  This is
because $x$ could contain a newline. In that case, we would have:

\[ \cat{\nl}{(\choice{y}{z})} =^? \choice{(\cat{\nl}{y})}{(\cat{\nl}{z})} \]

The document on the left makes a legitimate choice between $y$ and $z$, comparing their first lines
to see which one fits. However, the document on the right makes a degenerate choice: it compares the
first lines of $\cat{\nl}{y}$ and $\cat{\nl}{z}$, which are both empty! Thus it always picks
$\cat{\nl}{y}$, regardless of $y$ and $z$.  As a result, the concatenation of a newline and a choice
cannot be simplified.

For a similar reason, $\cat{(\choice{x}{y})}{(\choice{z}{w})}$ is not always equal to
$\choice{\choice{(\cat{x}{z})}{(\cat{y}{z})}}{\choice{(\cat{x}{w})}{(\cat{y}{w})}}$.

\subsection{Laws for Pretty Printing}

\paragraph{Resolving Choices}

\begin{Table}
$\resolve{w}{\nil}$ &$=$& $\nil$ \\
\end{Table}

\paragraph{Layout}

\begin{Table}
$\layout{i}{\nil}$ &$=$& \texttt{""} \\
$\layout{i}{\txt}$ &$=$& $\txt$ \\
$\layout{i}{\nl}$ &$=$& \texttt{"{\textbackslash}n"} $\doubleplus$ $i*$\texttt{" "} \\
$\layout{i}{\ind{j}{x}}$ &$=$& $\layout{i+j}{x}$ \\
$\layout{i}{\flat{x}}$ &$=$& $\layout{i}{x}$ \\
$\layout{i}{\cat{x}{y}}$ &$=$& $\layout{i}{x} \doubleplus \layout{i}{y}$ \\
\end{Table}

\paragraph{Pretty Printing}

\[ \pp{w}{x} = \layout{0}{\resolve{w}{x}} \]

\section{Normal Form}

[FILL]

\section{Implementation}

We can use these laws to derive an efficient implementation.

\end{document}
